\documentclass[12pt]{article}
\usepackage{sbc-template}
\usepackage{graphicx}
\usepackage{amsmath}
\usepackage{subfigure}
\usepackage{times,amsmath,epsfig}
\usepackage{graphicx,url}
  \makeatletter
  \newif\if@restonecol
  \makeatother
  \let\algorithm\relax
  \let\endalgorithm\relax
\usepackage[lined,algonl,ruled]{algorithm2e}
\usepackage{multirow}
\usepackage[brazil]{babel}
\usepackage[utf8]{inputenc}
\usepackage[pdftex]{hyperref}

\sloppy

\title{Redes de Computadores \\ Trabalho Prático 1 \\
\huge{Cliente e servidor em Sockets}}
\date{November 22, 1963}


\author{André Taiar Marinho Oliveira}


\address{Departamento de Ciência da Computação -- Universidade Federal de Minas Gerais (UFMG)
\email{taiar@dcc.ufmg.br}
\\
\\ Outubro de 2016
}

\begin{document}

\maketitle

\begin{resumo}
O problema tratado nesse trabalho é o de comunicação entre softwares cliente e
servidor através de sockets em redes de computadores. Para isso, foi definido um
protocolo de comunicação para um problema bem específico: coletar tempos de
atletas das olimpíadas e retornar em ordem de chegada ou individualmente (quando
passado o parâmetro da colocação do atleta). Foi feita a implementação dos dois
programas (cliente e servidor) de forma modularizada e um Makefile para orientar
a compilação do projeto.
\end{resumo}

\section{Implementação}

O trabalho foi implementado em C++ para que eu pudesse me aproveitar melhor dos
módulos de forma orientada a objetos. Foram implementados dois programas,
cliente e servidor, de forma modularizada (cada módulo é uma classe
basicamente).

Não foi necessária a implementação de nenhuma estrutura de dados mais elaborada.
Nas versões iniciais do trabalho eu havia implementado uma lista para colecionar
os tempos informados mas, quando eu precisei ordená-los isso se tornou inviável
e eu decidi usar o $vector$ do C++.

Os módulos implementados bem como suas funções são descritos abaixo.

\subsection{Token}

A classe Token é responsável pela leitura da entrada fornecida pelo cliente.
Basicamente ela separa a string fornecida por cada linha da entrada em
sub-strings de forma que possamos fazer a leitura dos inputs fornecidos. O
cabeçalho da classe é fornecido a seguir:

\begin{verbatim}
  #ifndef __TOKEN_H
  #define __TOKEN_H TOKEN

  #define DELIMITER " "

  #include <string>
  #include <vector>

  class Token {
  public:
    Token(std::string);
    std::string getToken();
    std::string getNextToken();
    unsigned int total();
    int getCurrent();

  private:
    std::string input;
    std::vector<std::string> tokens;
    int current;

    void tokenize();
  };

  #endif
\end{verbatim}

\subsection{Tempo}
A classe Tempo é responsável por encapsular todo o processamento que precisamos
fazer sobre o input de tempo no sistema. Suas principais funções são:

\begin{enumerate}
  \item Instanciar o tempo à partir de uma string fornecida pela entrada;
  \item Comparar se um tempo fornecido é maior ou menor que outro tempo.
\end{enumerate}

O cabeçalho da classe é fornecido a seguir:

\begin{verbatim}
  #ifndef __TEMPO_H
  #define __TEMPO_H TEMPO

  #include <iostream>
  #include <vector>
  #include <sstream>
  #include <string>

  #include <cctype>
  #include <cstdlib>

  #include "token.h"

  using namespace std;

  class Tempo {
  public:
    int hours;
    int minutes;
    int seconds;
    int milisseconds;

    Tempo();
    Tempo(int, int, int, int);
    int compare(Tempo);
    Tempo* setFromString(char[]);
    string toString();
    void print();
    bool biggerThan(Tempo*);
    bool smallerOrEqualThan(Tempo*);

  private:
    void parseUnit(string);
    bool isNumber(char);
    void setTimeUnit(string, string);
    int toPseudoMs();
  };

  #endif
\end{verbatim}

\subsection{Servidor}

A classe Servidor é responsável por encapsular toda a implementação do software
servidor e a utilização de sockets para isso. Também contém a implementação do
método $main$ da aplicação servidor. A intenção dela é abstrair as arestas
trazidas pelas bibliotecas de sockets em C e implementar métodos que possam ser
mais fáceis de usar.

Um bom exemplo de encapsulamento de um comportamento mais complicado seria o
método a função $Servidor::sendToClient$ que envia para o cliente já conectado
uma string qualquer com uma quebra de linha ao final. Toda a comunicação que é
feita dos servidor para o cliente neste trabalho utiliza esse método.

Uma outra vantagem em trabalhar com C++ neste projeto foi a maior facilidade de
lidar com as strings ao invés de cadeias de caracteres.

O cabeçalho da classe Servidor é fornecido a seguir:

\begin{verbatim}
  #ifndef __SERVIDOR_H
  #define __SERVIDOR_H SERVIDOR

  #include <vector>
  #include <string>
  #include <cstdlib>
  #include <cstring>
  #include <cstdio>
  #include <unistd.h>

  #include <sys/socket.h>
  #include <arpa/inet.h>

  #include "tempo.h"

  struct sockaddr;

  void logexit(const char*);
  void fill(const struct sockaddr *addr, char *line);

  class Servidor {
  public:
    Servidor(int);
    void logexit(const char*);
    void fill(const struct sockaddr*, char*);
    void run();
    void pushTime(char[]);
    void getPosition(char[]);
    void dumpTimes();
    void shutdown();
    static int compare(const void*, const void*);
    void parse(char[]);
    void sendToClient(string);

  private:
    int porta;
    int s;
    int r;
    std::vector<Tempo*> tempos;
    Tempo* returnThePosition(unsigned int);
  };

  #endif
\end{verbatim}

\subsection{Implementação do software cliente}

Apesar de ter a extensão $.cpp$ o código do cliente não é modularizado e está
escrito em C mesmo. Eu basicamente utilizei o exemplo mostrado pelo monitor
com algumas partes do exemplo publicado no Moodle e implementei o trecho de
comunicação com o servidor para que tudo funcionasse bem. O maior foco do
trabalho foi, com certeza, no software servidor.

\subsection{Decisões de implementação}

Uma decisão de implementação relevante fora o que já descrevi
anteriormente é que ao colecionar os tempos no vetor eu faço a inserção deles já
em ordem crescente. Dessa forma, quando recebo um novo tempo do cliente, eu
vou comparando com os tempos já inseridos desde o início do vetor. Ao encontrar
algum tempo que seja maior do que o que estou analisando, por exemplo, na
posição $N$, eu insiro o novo tempo nessa posição e desloco todos os outos já
existentes uma posição para frente no vetor.

Outra decisão, não tão baixo nível quanto a descrita anteriormente, é sobre as
respostas do servidor retornadas ao cliente. Para toda mensagem recebida na
servidor, o cliente espera uma resposta. Esta resposta pode ser para imprimir
algo, continuar esperando o próximo comando ou terminar a execução do cliente.
Esta implementação fica claro no software cliente neste trecho:

\begin{verbatim}
  if(strcmp(line2, "Z\n") == 0)
    break;
  else if(strcmp(line2, "O\n") == 0 || strcmp(line2, "D\n") == 0 || strcmp(line2, "C\n") == 0)
    continue;
  else {
    printf("%s", line2);
    continue;
  }
\end{verbatim}

\section{IPv4 e IPv6}

Para fazer funcionar em ambos protocolos, eu me baseei no exemplo de código
enviado no Moodle. Eu implementei no servidor o método $Servidor::fill$ para
que ele utilize corretamente o protocolo de acordo com o que o cliente tente se
conectar.

\section{Testes}
Segue alguns testes efetuados mostrando várias conexões não-simultâneas de
clientes ao servidor. A execução do servidor será omitida visto que ele não
faz qualquer output durante o seu funcionamento.

\begin{verbatim}
  ~ src git:(master) ./cliente 127.0.0.1 1234
  D 1h
  D 61m
  D 1m 1h 1ms
  O
  1h 0m 0s 0ms
  0h 61m 0s 0ms
  1h 1m 0s 1ms
  D 2h
  O
  1h 0m 0s 0ms
  0h 61m 0s 0ms
  1h 1m 0s 1ms
  2h 0m 0s 0ms
  C 2
  0h 61m 0s 0ms
  C 7
  C 0
  C -1
  C 1
  1h 0m 0s 0ms
  C 4
  2h 0m 0s 0ms
  O
  1h 0m 0s 0ms
  0h 61m 0s 0ms
  1h 1m 0s 1ms
  2h 0m 0s 0ms
  Z
  ~  src git:(master) ./cliente 127.0.0.1 1234
  O
  1h 0m 0s 0ms
  0h 61m 0s 0ms
  1h 1m 0s 1ms
  2h 0m 0s 0ms
  D 22m
  O
  0h 22m 0s 0ms
  1h 0m 0s 0ms
  0h 61m 0s 0ms
  1h 1m 0s 1ms
  2h 0m 0s 0ms
  Z
  ~  src git:(master)
\end{verbatim}

\section{Conclusão}

Eu gostei de trabalhar na implementação deste trabalho. Já tinha programado em
redes mas com linguagens mais alto nível (basicamente em Ruby) e ver como
funciona a implementação de trocas de mensagem foi bem interessante.

Acredito que tenha acertado na modularização do meu trabalho visto que posso
aproveitar a arquitetura produzida para os próximos trabalhos que seguirem a
mesma linha de problema.

\end{document}
